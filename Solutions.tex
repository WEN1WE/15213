\documentclass{report}
\usepackage[utf8]{inputenc}
 
\usepackage{graphicx} %package to manage images
\usepackage[rightcaption]{sidecap}
\usepackage{wrapfig}
\usepackage{float}
 
\usepackage{listings}
\usepackage{color}
 
\definecolor{codegreen}{rgb}{0,0.6,0}
\definecolor{codegray}{rgb}{0.5,0.5,0.5}
\definecolor{codepurple}{rgb}{0.58,0,0.82}
\definecolor{backcolour}{rgb}{0.95,0.95,0.92}
 
\lstdefinestyle{mystyle}{
    backgroundcolor=\color{backcolour},   
    commentstyle=\color{codegreen},
    keywordstyle=\color{magenta},
    numberstyle=\tiny\color{codegray},
    stringstyle=\color{codepurple},
    basicstyle=\footnotesize,
    breakatwhitespace=false,         
    breaklines=true,                 
    captionpos=b,                    
    keepspaces=true,                 
    numbers=left,                    
    numbersep=5pt,                  
    showspaces=false,                
    showstringspaces=false,
    showtabs=false,                  
    tabsize=2
}
 
\lstset{style=mystyle}

\begin{document}

\chapter{A Tour of Computer Systems}

\chapter{Representing and Manipulating Information}

\section{Information Storage}
\paragraph{Solution to Problem } 2.1 \\
A: 0011 1001 1010 0111 1111 1000 \\
B: 0xC97B  \\
C: 1101 0101 1110 0100 1100  \\
D: 0x26E7B5  \\

\paragraph{Solution to  Problem } 2.2
\begin{center}
\begin{tabular}{ |c|c|c| } 
\hline
n & \(2^n\)(decimal) & \(2^n\)(hexadecimal) \\
\hline\hline
9 & 512 & 0x200 \\
\hline
19 & 524288 & 0x80000 \\
\hline
14 & 16384 & 0x4000 \\
\hline
16 & 65536 & 0x10000 \\
\hline
17 & 131072 & 0x20000 \\
\hline
5 & 32 & 0x20  \\
\hline
7 & 128 & 0x80 \\
\hline 
\end{tabular}
\end{center}

\paragraph{Solution to Problem} 2.3
\begin{center}
\begin{tabular}{ |c|c|c| } 
\hline
Decimal & Binary & Hexadecimal \\
\hline\hline
0 & 0000 0000 & 0x00 \\
\hline
167 & 1010 0111 & 0xA7 \\
\hline
62 &  0011 1110 & 0x3E \\
\hline
188 & 1011 1100 & 0xBC  \\
\hline
55 & 0011 0111 & 0x37  \\
\hline
136 & 1000 1000 & 0x88 \\
\hline
243 & 1111 0011 & 0xF3 \\
\hline
82 & 0101 0010 & 0x52 \\
\hline
172 & 1010 1100 & 0xAC \\
\hline
231 & 1110 0111 & 0xE7  \\
\hline
\end{tabular}
\end{center}

\paragraph{Solution to Problem} 2.3 \\
A: 0x5044  \\
B: 0x4FFC  \\ 
C: 0x507C  \\
D: 0xAE  \\

\paragraph{Code} 2-4

\begin{lstlisting}

#include <stdio.h>
typedef unsigned char *byte_pointer;

void show_bytes(byte_pointer start, size_t len) {
	size_t i;
	for (i = 0; i < len; i++)
		printf("%.2x", start[i]);
	printf("\n");
}

void show_int(int x) {
	show_bytes((byte_pointer)&x, sizeof(int));
}

void show_float(float x) {
	show_bytes((byte_pointer)&x, sizeof(float));
}

void show_pointer(void *x) {
	show_bytes((byte_pointer)&x, sizeof(void *));
}

void test_show_bytes(int val) {
	int ival = val;
	float fval = (float)ival;
	int *pval = &ival;
	show_int(ival);
	show_float(fval);
	show_pointer(pval);
}

int main() {
	test_show_bytes(12345);
	return 0;
}

\end{lstlisting}
\includegraphics[scale=0.4]{Solution/images/Code2_4.png}

\paragraph{Solution to Problem} 2.5 \\
A: Little endian:21 \hspace{10mm} Big endian:87 \\
B: Little endian:21 43 \hspace{6mm} Big endian:87 65 \\
C: Little endian:21 43 65 \hspace{1mm} Big endian 87 65 43

\paragraph{Solution to Problem} 2.6  \\
A: 0x00359141 : 00000000001 101011001000101000001 

0x4A564504 : \hspace{1mm} 010010100 101011001000101000001 00 \\
B: 21 \\
C: We find all bits of the integer embedded in the floating-point number, except for the most significant bit aving value 1.

\paragraph{Solution to Problem} 2.7 \\
61 62 63 64 65 66(strlen does not count the terminating null character.

\paragraph{Solution to Problem} 2.9 \\
A: white(black) yello(bule) magenta(green) cyan(red) \\
B: Cyan(011) Green(010) Blue(001)

\paragraph{Solution to Problem} 2.10
\begin{center}
\begin{tabular}{ |c|c|c| } 
\hline
Step & *x & *y \\
\hline\hline
Initially & a & b \\
\hline
Step 1 & a & a\textasciicircum b \\   
\hline  
Step 2 & b & a\textasciicircum b \\
\hline
Step 3 & b & a \\
\hline
\end{tabular}
\end{center}

\paragraph{Solution to Problem} 2.11 \\
A: Both first and last have value k. \\
B: In this case, x and y point to the same location, and a\textasciicircum a = 0 \\
C: In line 4, replace first $\leq$ last to  first $<$ last

\paragraph{Solution to Problem} 2.12 \\
A: x $\&$ 0xFF  \\
B: x\textasciicircum {\textasciitilde 0xFF} \\
C: x \textbar \hspace{1mm} 0xFF \\

\paragraph{Solution to Problem} 2.13 \\
\begin{lstlisting}
int bis(int x, int m) {
	return x | m;
}

int bic(int x, int m) {
	return x & (!m);
}

int bool_or(int x, int y) {
	int result = bis(x, y);
	return result;
}

int bool_xor(int x, int y) {
	int result = bis(bic(x, y), bic(y, x));
	return result;
}

\end{lstlisting}

\paragraph{Solution to Problem} 2.14 \\
\begin{center}
\begin{tabular}{ |c|c|c| } 
\hline
logical operators & bit level operations   \\
\hline\hline  
$\&\&$ & $\&$  \\
\hline  
\textbar \textbar & \textbar  \\
\hline
$!$  & \textasciitilde  \\
\hline
& \textasciicircum \\
\hline
\end{tabular}
\end{center}

\paragraph{Solution to Problem} 2.15 \\
\begin{lstlisting}
!(x ^ y)
\end{lstlisting}


\paragraph{Solution to Problem} 2.16 \\
\begin{center}
\begin{tabular}{ |c|c|c|c|c|c|c|c| }  
\hline
Hex & Binary & Binary & Hex & Binary & Hex & Binary & Hex  \\
\hline\hline
0xC3 & [11000011] &  [00011000] & 0x18 & [00110000] & 0x30 & [11110000] & 0xF0 \\
\hline
0x75 & [01110101] &  [10101000] & 0xA8 & [00011101] & 0x1D &[00011101] & 0x1D \\ 
\hline
0x87 & [10000111] &  [00111000] & 0x38 & [00100001] & 0x21 & [11100001] & 0xE1 \\
\hline
0x66 & [01100110] &  [00110000] & 0x30 & [00011001] & 0x19 & [00011001] & 0x19 \\
\hline
\end{tabular}
\end{center}

\paragraph{Solution to Problem} 2.17 \\

\begin{center}
\begin{tabular}{ |c|c|c|c| } 
\hline
0xE & [1110] & 14 & -2 \\
\hline
0x0 & [0000] & 0 & 0 \\
\hline
0x5 & [0101] & 5 & 5 \\
\hline
0x8 & [1000] & 8 & 8 \\
\hline
0xD & [1101] & 13 & -3 \\
\hline
0xF & [1111] & 15 & -1 \\
\hline
\end{tabular}
\end{center}

\paragraph{Solution to Problem} 2.19 \\
\begin{center}
\begin{tabular}{ |c|c|c| } 
\hline
x & $T2U_4(x)$ \\
\hline
-8 & 8 \\
-3 & 13 \\
-2 & 14 \\
-1 & 15 \\
0 & 0 \\
5 & 5 \\
\hline
\end{tabular}
\end{center}

\paragraph{Solution to Problem} 2.21 \\

\begin{center}
\begin{tabular}{ |c|c|c| } 
\hline
Expression & Type & Evaluation \\
\hline
& Unsighed & 1 \\
\hline
& Signed & 1 \\
\hline
& Unsighed & 0 \\
\hline
& Sighed & 1 \\
\hline
& Unsigned & 1 \\
\hline
\end{tabular}
\end{center}

\paragraph{Solution to Problem} 2.23 \\
\begin{center}
\begin{tabular}{ |c|c|c| } 
\hline
w & fun1(w) & fun2(w) \\
\hline
& 0x00000076 & 0x00000076 \\
\hline
& 0x00000021 & 0x00000021 \\
\hline
& 0x000000C9 & 0xFFFFFFC9 \\
\hline
& 0x00000087 & 0xFFFFFF87 \\
\hline
\end{tabular}
\end{center}

\paragraph{Solution to Problem} 2.25 \\
Since parameter length is unsigned, the computation 0 - 1 performed using unsigned arithmetic, so the result is then UMax.
\begin{lstlisting}
// line 6  
for (i = 0; i < length; i++)
\end{lstlisting}

\paragraph{Solution to Problem} 2.26 \\
A: The function will incorrectly return 1 when s is shorter than t.
B: Since strlen is defined to yield an unsigned result, the difference and the comparison are both computed using unsigned arithmetic.
C:
\begin{lstlisting}
return strlen(s) > strlen(t);
\end{lstlisting}

\paragraph{Solution to Problem} 2.27 \\
\begin{lstlisting}
return  x + y > x
\end{lstlisting}

\paragraph{Solution to Problem} 2.28 \\
\begin{center}
\begin{tabular}{ |c|c|c|c| } 
\hline
\multicolumn{2}{|c|}{x}  &   \multicolumn{2}{|c|}{$-^u_4$x}  \\  \hline
Hex & Decimal & Decimal & Hex  \\   \hline 
0 & 0 & 0 & 0 \\
5 & 5 & 11& B  \\ 
8 & 8 & 8 & 8 \\
D & 13 & 3 & 3 \\
F & 15 & 1 & 1 \\ \hline
\end{tabular}
\end{center}

\paragraph{Solution to Problem} 2.30 \\


\begin{lstlisting}
int tadd_ok(int x, int y) {
    int sum = x + y;
    int neg_over = x < 0 && y < 0 && sum >= 0;
    int pos_over = x >= 0 && y >= 0 && sum < 0;
    return !neg_over && !pos_over;
}
\end{lstlisting}

\paragraph{Solution to Problem} 2.31 \\
The expression $(x+y)-x$ will evaluate to y regardless of whether or not the addition overflows.

\paragraph{Solution to Problem} 2.32 \\
When y is TMin, we will have -y also equal to TMin.


\paragraph{Solution to Problem} 2.33 \\
\begin{center}
\begin{tabular}{ |c|c|c|c| } 
\hline
\multicolumn{2}{|c|}{x}  &   \multicolumn{2}{|c|}{$-^t_4$x}  \\  \hline
Hex & Decimal & Decimal & Hex  \\   \hline 
0 & 0 & 0 & 0 \\
5 & 5 & -5 & B  \\ 
8 & -8 & -8 & 8 \\
D & -3 & 3 & 3 \\
F & -1 & 1 & 1 \\ \hline
\end{tabular}
\end{center}

\paragraph{Solution to Problem} 2.35 \\
1: x,y can be written as a 2w-bit two's-complement number.Let u denote the unsigned number represented by the lower w bits, and v denote the two's-complement number represented by the upper w bits. \\
$x * y = v2^w + u$ \\
$u = p + p_(w-1)2^w$ \\
let $t = v + p_(w-1)$
so $x * y = p + t2^w$ \\
When $t = 0$, we have $x * y = p$; the multiplication does not overflow.When $t \neq 0$ ,we have $x * y \neq p$;the multiplication does overflow.



\paragraph{Solution to Problem} 2.36 \\


\begin{lstlisting}
int tmult_ok(int x, int y) {
    int64_t pll = (int64_t) x*y
    
    return pll == (int) pll;
}
\end{lstlisting}

\paragraph{Solution to Problem} 2.38 \\
We can compute multiples 1,2,3,4,5,8,9

\paragraph{Solution to Problem} 2.39 \\
The expression simply becomes $-(x\ll m)$

\paragraph{Solution to Problem} 2.40 \\
\begin{center}
\begin{tabular}{ |c|c|c|c| } 
\hline
K & shifts & Add/Subs & Expression \\ \hline
6 & 2 & 1 & $(x \ll 2) + (x \ll 1)$ \\
31 & 1 & 1 & $(x \ll 5) - x$  \\
-6 & 2 & 1 & $(x \ll 1) - (x \ll 3)$  \\
55 & 2 & 2 & $(x \ll 6) - (x \ll 3)$  \\ \hline
\end{tabular}
\end{center}

\paragraph{Solution to Problem} 2.41 \\
Assuming that addition and subtraction have the same performance, the rule is to choose form A when $n = m$, either form when $n = m + 1$, and form B, when $n > m + 1$

\paragraph{Solution to Problem} 2.42 \\
\begin{lstlisting}
int div16(int x) {
    int bias = (x >> 31) & 0xF;
    return (x + bias) >> 4;
}
\end{lstlisting}


\paragraph{Solution to Problem} 2.43 \\
M is 31 and N is 8.

\paragraph{Solution to Problem} 2.44 \\
A: False. Let x be -2147483648($TMin_32$). \\
B: True \\
C: False. Let x be 65535. \\
D: True \\
E: False. Let x be $-2^{31}$
F: True. Two's-complement and unsigned addition have the same bit-level behavior, and thet are commutative. \\
G: True. \textasciitilde y = -y - 1

\paragraph{Solution to Problem} 2.46 \\
A: 0.0000 0000 0000 0000 0000 000 1100[1100]...

\paragraph{Solution to Problem} 2.47 \\
\begin{center}
\begin{tabular}{ |c|c|c|c|c|c|c|c|c| } 
\hline
 Bits & e & E & $2^E$ & f & M & $2^E x M$ & V & Decimal \\
0 00 00 & 0 & 0 & 1 & 0/4 & 0/4 & 0/4 & 0 & 0.0 \\
0 00 01 & 0 & 0 & 1 & 1/4 & 1/4 & 1/4 & 1/4 & 0.25 \\
0 00 10 & 0 & 0 & 1 & 2/4 & 2/4 & 2/4 & 2/4 & 0.5 \\
0 00 11 & 0 & 0 & 1 & 3/4 & 3/4 & 3/4 & 3/4 & 0.75 \\
0 01 00 & 1 & 0 & 1 & 0/4 & 4/4 & 4/4 & 1 & 1.1 \\
0 01 01 & 1 & 0 & 1 & 1/4 & 5/4 & 5/4 & 5/4 & 1.25 \\
0 01 10 & 1 & 0 & 1 & 2/4 & 6/4 & 6/4 & 3/2 & 1.5 \\
0 01 11 & 1 & 0 & 1 & 3/4 & 7/4 & 7/4 & 7/4 & 1.75 \\
0 10 00 & 2 & 1 & 2 & 0/4 & 4/4 & 8/4 & 2 & 2.0 \\
0 10 01 & 2 & 1 & 2 & 1/4 & 5/4 & 10/4 & 5/2 & 2.5 \\
0 10 10 & 2 & 1 & 2 & 2/4 & 6/4 & 12/4 & 3 & 3.0 \\
0 10 11 & 2 & 1 & 2 & 3/4 & 7/4 & 14/4 & 7/2 & 3.5 \\
\hline
\end{tabular}
\end{center}


\paragraph{Solution to Problem} 2.49 \\
A: The number has binary representation 1, followed by n zeros, followed by1, giving value $2^{n+1} + 1$. \\
B: When n = 23, the value is $2^{24} + 1 = 16777217$

\paragraph{Solution to Problem} 2.50 \\
A: 10.0
B: 10.1
C: 11.0
D: 11.0

\paragraph{Solution to Problem} 2.51 \\
\begin{center}
\begin{tabular}{ |c|c|c|c| } 
\hline
\multicolumn{2}{|c|}{Format A}  &   \multicolumn{2}{|c|}{Format B}  \\  
\hline
Bits & Value & Bits & Value  \\ 
\hline
011 0000 & 1 & 0111 000 & 1 \\
101 1110 & 15/2 & 1001 111 & 15/2 \\
010 1001 & 25/32 & 0110 100 & 3/4 \\
110 1111 & 31/2 & 1011 000 & 16 \\
000 0001 & 1/64 & 0001 000 & 1/64 \\
\hline
\end{tabular}
\end{center}


\chapter{Machine-Level Representation of Programs}
\paragraph{Solution to Problem } 3.1 \\
\begin{center}
\begin{tabular}{ |c|c| } 
\hline
Operand & Value \\
\hline
 & 0x100 \\
 & 0xAB \\
 & 0x108 \\
 & 0xFF \\
 & 0xAB \\
 & 0x11 \\
 & 0x13 \\
 & 0xFF \\
 & 0x11 \\
\hline
\end{tabular}
\end{center}

\paragraph{Solution to Problem } 3.2 \\
\begin{lstlisting}
movl  %eax, (%rsp)
movw (%rax), %dx
movb $0xFF, %bl
movb (%rsp, %rdx, 4), %dl
movq (%rdx), %rax
movw %dx, (%rax)

\end{lstlisting}

\paragraph{Solution to Problem } 3.3 \\

\begin{lstlisting}
movb $0xF, (%ebx)  Cannot use %ebx as address register
movl %rax, (%rsp)  Mismatch between instruction suffix and register ID
movw (%rax), 4(%rsp) Cannot have both source and destination be memory references
movb %al, %sl No register named %sl
movl %eax, %dx Destination operand incorrect size
movb %si, 8(%rbp) Mismatch betwen instruction suffix and register ID

\end{lstlisting}


\paragraph{Solution to Problem } 3.4 \\
\begin{lstlisting}
movsbl (%rdi), %eax  
movl %eax, (%rsi)

movsbl (%rdi), %eax
movl %eax, (%rsi)

movzbl ($rdi), %eax
movq %rax, (%rsi)

movl (%rdi), %eax
movb %al, (%rsi)

movl (%rdi), %eax
movb %al, (%rsi)

movl(%rdi), %eax
movb %al, (%rsi)

movsbw (%rdi), %ax
movw %ax, (%rsi)
\end{lstlisting}

\paragraph{Solution to Problem } 3.5 \\
\begin{lstlisting}
void decode1(long *xp, long *yp, long *zp) {
    long x = *xp;
    long y = *yp;
    long z = *zp;
    *yp = x;
    *zp = y;
    *xp = z;
    
    return z;

}
\end{lstlisting}

\paragraph{Solution to Problem } 3.6 \\
\begin{center}
\begin{tabular}{ |c|c| } 
\hline
Instruction & Result \\
 & 6 + x \\
 & x + y \\
 & x + 4y \\
 & 9x + 7 \\
 & 4y + 10 \\
 & x + 2y + 9 \\
\hline
\end{tabular}
\end{center}

\paragraph{Solution to Problem } 3.7 \\
\begin{lstlisting}
long x = 5 * x + 2 * y + 8 * z;
\end{lstlisting}

\paragraph{Solution to Problem } 3.8 \\
\begin{center}
\begin{tabular}{ |c|c|c| } 
\hline
Instruction & Destination & Value \\
 & 0x100 & 0x100 \\
 & 0x108 & 0xA8 \\
 & 0x118 & 0x110 \\
 & 0x110 & 0x14 \\
 & \%rcx & 0x0 \\
 & \%rax & 0xFD \\
\hline
\end{tabular}
\end{center}

\paragraph{Solution to Problem } 3.9 \\
\begin{lstlisting}
salq $4, %rax 

sarq %cl, %rax 

\end{lstlisting}

\paragraph{Solution to Problem } 3.10 \\
\begin{lstlisting}
long t1 = x | y;
long t2 = t1 >> 3;
long t3 = ~t2;
long t4 = z-t3;
\end{lstlisting}

\paragraph{Solution to Problem } 3.11 \\
A. This instruction is used to set register \%rdx to zero. \\
B. \$0, \%rdx \\
C. We find that the version with xorq requires only 3 bytes,while the version with movq requires 7.Other way to set \%rdx to zero rely on the property that any instruction that updates the lower 4 bytes will cause the high-order bytes to be set zero.Thus. we could use either xorl \%edx, \%edx(2 bytes) or movl \$0, \%edx(5 bytes)


\paragraph{Solution to Problem } 3.12 \\
\begin{lstlisting}
uremdiv:
    movq    %rdx, %r8  
    movq    %rdi, %rax
    movl    $0, %edx
    divq    %rsi
    movq    %rax, (%r8)
    movq    %rdx, (%rcx)
    ret
    
\end{lstlisting}


\paragraph{Solution to Problem } 3.13 \\
A. The suffix'1' and the register identifiers 32-bit operands,while the comparision is for a two's-complement $<$, We can infer that data\_t must be int. \\
B. The comparison is for a two's-complement $>=$ , and data\_t must be short. \\
C. The comparison is for an unsigned $<=$, and data\_t must be unsigned char\\
D. The suffix 'q' and the register identifiers indicate 8-bit operands, while the comparison is for $!=$. We can infer that data\_t could be either long, unsigned long , or some form of pointer.

\paragraph{Solution to Problem } 3.14 \\
A. The comparison if for a two's-complement $>=$, and data\_t must be long. \\
B. The comparison is for $==$, which is the same for signed or unsigned, and data\_t must be either short or unsigned short. \\
C. The comparison is for unsigned $>$, and data\_t must be unsigned char. \\
D. The comparison is for $!=$, which is the same for signed or unsigned, and data\_t could be int or unsigned int.

\paragraph{Solution to Problem } 3.15 \\
A. The je instruction has as its target 0x4003fc + 0x02:0x4003fe. \\
B. The je instruction has as its target 0x400431 - 12:0x400425. \\
C. ja:0x400543, pop:0x400545 \\
D. The jmpq instruction has as its target 0x400560

\paragraph{Solution to Problem } 3.16 \\
A. 
\begin{lstlisting}
void goto_cond(long a, long*p) {
    if (p == 0)
        goto done;
    if (*p >= a) 
        goto done;
    *p = a;
done:
    return;
}
\end{lstlisting}
B. The first conditional branch is part of the implementation of the $\&\&$ expression. If the test for p being non-null fails, the code will skip the test of $a > *p$

\paragraph{Solution to Problem } 3.17 \\
A.
\begin{lstlisting}
long gotodiff_se_alt(long x, long y) {
    long result;
    if (x < y) 
        goto x_lt_y;
    ge_cnt++;
    result = x - y;
x_lt_y:
    lt_cnt++;
    result = y - x;
    return result;
}
\end{lstlisting} 
B. In most respects, the choice is arbitrary. But the original rule works better for the common case where there is no else statement.


\paragraph{Solution to Problem } 3.18 \\
\begin{lstlisting}
long test(long x, long y, long z) {
    long val = x + y + z;
    if (x < -3) {
        if (y < z) 
            val = x * y;
        else
            val = y * z;
    } else if (x > 2) 
        val = x * z;
    
    return val;
}
\end{lstlisting}

\paragraph{Solution to Problem } 3.19 \\
A. $T_{MP} = 2(31-16) = 30$ \\
B. 46

\paragraph{Solution to Problem } 3.20 \\
A. The operator is '/' \\
B.
\begin{lstlisting}
arith:
    leaq    7(%rdi), %rax   temp = x + 7
    testq   %rdi, %rdi      Test x
    cmovns

%rdi, %rax      If x >= 0, temp = x
    sarq    $3, %rax        result = temp >> 3(= x/8)
    ret
    
\end{lstlisting}

\paragraph{Solution to Problem } 3.21 \\
\begin{lstlisting}
long test(long x, long y) {
    long val = 8 * x;
    if (y > 0) {
        if (x < y) 
            val = y - x;
        else
            val = x & y;
    } else if (y <= -2) 
        val = x + y;
    return val;
}
\end{lstlisting}

\paragraph{Solution to Problem } 3.23 \\
A. \%rax, \%rcs, \%rdx are initialized in lines 2-5 to x, x*x and x+x \\
B. The compiler determines that pointer p always points to x, and hence the expression (*p)++ simply increments x. It combines this incrementing by 1 with the increment by y, via the leaq instruction of line 7. \\
C. 
\begin{lstlisting}
dw_loop:
    movq    %rdi, %rax      Copy x to %rax
    movq    %rdi, %rcx
    imulq   %rdi, %rcx
    leaq    (%rdi, %rdi), %rdx  Compute n = 2*n
.L2:
    leaq    1(%rcx, %rax), %rax  Compute x += y + 1
    subq    $1, %rdx
    testq   %rdx, %rdx      Test n
    jq      .L2             If > 0, goto loop
    rep; ret
    
\end{lstlisting}

\paragraph{Solution to Problem } 3.24 \\
\begin{lstlisting}
long loop_while(long a, long b) {
    long result = 1;
    while (a < b) {
        result = result * (a + b);
        a = a + 1;
    }
    
    return result;
}
\end{lstlisting}

\paragraph{Solution to Problem } 3.25 \\
\begin{lstlisting}
long loop_while2(long a, long b) {
    long result = b;
    while (b > 0) {
        result = result * a;
        b = b - a;
    }
}
\end{lstlisting}

\paragraph{Solution to Problem } 3.26 \\
A. We can see that the code uses the jump-to-middle translation, using the jmp instruction on line 3. \\
B.
\begin{lstlisting}
long fun_a(unsigned long x) {
    long val = 0;
    while (x) {
        val ^= x;
        x >>= 1;
    }
    return val & 0x1;
}
\end{lstlisting}
C.This code computes the parity of argument x. That is, it returns 1 if there is an odd number of ones in x and 0 if there is an even number.

\paragraph{Solution to Problem } 3.27 \\
\begin{lstlisting}
long fact_for_gd_goto(long n) {
    long i = 2;
    long result = 1;
    if (n <= 1) 
        goto done;
loop:
    result *= i;
    i++;
    if (i <= n)
        goto loop;
done:
    return result;
}
\end{lstlisting}

\paragraph{Solution to Problem } 3.28 \\
A. 
\begin{lstlisting}
long fun_b(unsigned long x) {
    long val = 0;
    long i;
    
    for (i = 64; i != 0; i--) {
        val = (val << 1) | (x & 0x1);
        x >>= 1;
    }
    return val;
}
\end{lstlisting}
B. The code was generated using the guarded-do transformation. \\
C. This code reverses the bits in x, creating a mirror image.

\paragraph{Solution to Problem } 3.29 \\
A. 
\begin{lstlisting}
long sum = 0;
long i = 0;
while (i < 10) {
    if (i & 1) 
        continue;
    sum += 1;
    i++;
}
\end{lstlisting}
This code has an infinite loop. \\
B.
\begin{lstlisting}
long sum = 0;
long i = 0;
while (i < 10) {
    if (i & 1)
        goto update;
    sum += 1;
    update:
        i++;
}
\end{lstlisting}

\paragraph{Solution to Problem } 3.30 \\
A. The case labels in the switch statement body have values -1, 0, 1, 2, 4, 5, 7. \\
B. The case with destination .L5 has labels 0 and 7. \\
C. The case with destination .L7 has labels 2 and 4.

\paragraph{Solution to Problem } 3.31 \\
\begin{lstlisting}
void switcher(long a, long b, long c, long *dest) {
    long val;
    switch(a) {
    case 5:
        c = b ^ 15;
        /* Fall through */
    case 0:
        val = c + 112;
        break;
    case 2:
    case 7:
        val = (c + b) << 2;
        break;
    case 4:
        val = a;
        break;
    default:
        val = b;
    }
    *dest = val;
}
\end{lstlisting}


\paragraph{Solution to Problem } 3.32 \\
\begin{center}
\begin{tabular}{ |c|c|c|c|c|c|c|c|c| } 
\hline
Label & PC & Instruction & \%rdi & \%rsi & \%rax & \%rsp &*\%rsp & Description \\
\hline
M1 & 0x400560 & callq & 10 & - & - & 0x7fffffffe820 & - & call first \\
F1 & 0x400548 & lea & 10 & - & - &  0x7fffffffe818 & 0x400565 & entry fisrt \\
F2 & 0x40054c & sub & 10 & 11 & - & 0x7fffffffe818 & 0x400565 & \\
F3 & 0x400550 & callq & 9 & 11 & - & 0x7fffffffe818 & 0x400565 & Call last \\
L1 & 0x400540 & mov & 9 & 11 & - &  0x7fffffffe810 & 0x400555 & Entry of last \\
L2 & 0x400543 & imul & 9 & 11 & 9 & 0x7fffffffe810 & 0x400555 & \\
L3 & 0x400549 & retq & 9 & 11 & 99 & 0x7fffffffe810 & 0x400555 & Return 99 from last \\
F4 & 0x400555 & repz repq & 9 & 11 & 99 & 0x7fffffffe818 & 0x400565 & Return 99 from first \\
M2 & 0x400565 & mov & 9 & 11 & 99 & 0x7fffffffe820 & - & Return main \\
\hline
\end{tabular}
\end{center}

\paragraph{Solution to Problem } 3.34\\
A. \%rbx, \%r15, \%r14, \%r13, \%r12, and \%rbp.\\
B. a6, a7.
C. After storing six local variables, the program has used up the supply of callee-saved registers. It stores the remaining two local values on the stack.

\paragraph{Solution to Problem } 3.35\\
A. Register \%rbx holds the value of parameter x. \\
B. 
\begin{lstlisting}
long rfun(unsigned long x) {
    if (x == 0) 
        return 0;
    unsigned long nx = x >> 2;
    long rv = rfun(nx);
    return x + rv;
}
\end{lstlisting}

\paragraph{Solution to Problem } 3.36\\
\begin{center}
\begin{tabular}{ |c|c|c|c|c|} 
\hline
Array & Element size & Total size & Start address & Element i \\ \hline
S & 2 & 14 & $x_s$ & $x_s + 2i$ \\
T & 8 & 24 & $x_T$ & $x_T + 8i$ \\
U & 8 & 48 & $x_u$ & $x_u + 8i$ \\
V & 4 & 32 & $x_v$ & $x_v + 4i$ \\
W & 8 & 32 & $x_w$ & $x_w + 8i$ \\
\hline
\end{tabular}
\end{center}

\paragraph{Solution to Problem } 3.37\\
\begin{center}
\begin{tabular}{ |c|c|c|c| } 
\hline
Expression & Type & Value & Assembly \\ \hline
S+1 & short * & $x_s + 2$ & leaq 2(\%rdx), \%rax \\
S[3] & short & M[$x_s+6$] & movw 6(\%rdx), \%ax \\ 
\&S[i] & short * & $x_s+2i$ & leaq (\%rdx, \%rcx, 2), \%rax \\
S[4*i+1] & short *& M[$x_s+8i+2$] & movw 2(\%rdx, \%rcx, 8), \%ax \\
S+i-5 & short * & $x_s+2i-10$ & leaq-10(\%rdx, \%rcx, 2), \%rax \\
\hline
\end{tabular}
\end{center}

\paragraph{Solution to Problem } 3.38\\
M = 5 \\
N = 7

\paragraph{Solution to Problem } 3.40\\
\begin{lstlisting}
void fix_set_diag_opt(fix_matrix A, int val) {
    int *Abase = &A[0][0];
    long i = 0;
    long iend = N*(N+1);
    do {
        Abase[i] = val;
        i += (N+1);
    } while (i != iend);
}
\end{lstlisting}

\paragraph{Solution to Problem } 3.41\\
A. p:0, s.x:8, s.y:12, next:16 \\
B. 24 bytes \\
C. 
\begin{lstlisting}
void sp_init(struct prob *sp) {
    sp->s.x = sp->s.y;
    sp->p = &(sp->s.x);
    sp->next = sp;
}
\end{lstlisting}

\paragraph{Solution to Problem } 3.42\\ 
A.
\begin{lstlisting}
long fun(struct ELE *ptr) {
    long val = 0;
    while (ptr) {
        val += ptr->v;
        ptr = ptr->p;
    }
    return val;
}
\end{lstlisting}
B. We can see that each structure is an element of a singly link list, with field v being the value of the element and p being a pointer to the next element.Function fun computes the sum of the element values in the list.

\paragraph{Solution to Problem } 3.43\\ 
\begin{center}
\begin{tabular}{ |c|c|c| } 
\hline
Expr & Type & Code \\
up $\rightarrow$ t1.u & long & movq (\%rdi), \%rax \\
& & movq \%rax, (\%rsi) \\ \hline
up $\rightarrow$ t1.v & short & movw 8(\%rdi), \%ax \\
& & movq \%rax, (\%rsi) \\ \hline 
\&up $\rightarrow$ t1.w & char * & addq \$10, \%rdi \\
& & movq \%rdi, (\%rsi) \\ \hline
up $\rightarrow$ t2.a & int * & movq \%rdi, (\%rsi) \\ \hline
up $\rightarrow$ t2.a[up $\rightarrow$ t1.u] & int & movq (\%rdi), \%rax \\
& & movl (\%rdi, \%rax, 4), \%eax \\
& & movl \%eax, (\%rsi) \\ \hline
*up $\rightarrow$ t2.p & char & movq 8(\%rdi), \%rax \\
& & movb (\%rax), \%al \\
& & movb \%al, (\%rsi) \\ \hline
\end{tabular}
\end{center}

\paragraph{Solution to Problem } 3.44\\
A. 
\begin{center}
\begin{tabular}{ |c|c|c|c|c|c| } 
\hline
i & c & j & d & Total & Alignment \\ \hline
0 & 4 & 8 & 12 & 16 & 4 \\
\hline
\end{tabular}
\end{center}
B. 
\begin{center}
\begin{tabular}{ |c|c|c|c|c|c| } 
\hline
i & c & d & j & Total & Alignment \\ \hline
0 & 4 & 5 & 8 & 16 & 8 \\
\hline
\end{tabular}
\end{center}
C.
\begin{center}
\begin{tabular}{ |c|c|c|c| } 
\hline
w & c & Total & Alignment \\ \hline
0 & 6 & 10 & 2  \\
\hline
\end{tabular}
\end{center}
D.
\begin{center}
\begin{tabular}{ |c|c|c|c| } 
\hline
w & c & Total & Alignment \\ \hline
0 & 16 & 40 & 8  \\
\hline
\end{tabular}
\end{center}
E.
\begin{center}
\begin{tabular}{ |c|c|c|c| } 
\hline
a & t & Total & Alignment \\ \hline
0 & 24 & 40 & 8  \\
\hline
\end{tabular}
\end{center}

\paragraph{Solution to Problem } 3.45\\
A. 
\begin{center}
\begin{tabular}{ |c|c|c|c|c|c|c|c|c| } 
\hline
Field & a & b & c & d & e & f & g & h \\ \hline
Size & 8 & 2 & 8 & 1 & 4 & 1 & 8 & 4 \\
Offset & 0 & 8 & 16 & 24 & 28 & 32 & 40 & 48 \\ 
\hline
\end{tabular}
\end{center}
B. The structure is a total of 56 bytes long. \\
C. 
\begin{lstlisting}
struct {
    char *a;
    double c;
    long g;
    float e;
    int h;
    short b;
    char d;
    char f;
} rec;
\end{lstlisting}

\paragraph{Solution to Problem } 3.46\\
A. 
\begin{center}
\begin{tabular}{ |c|c  } 
\hline
00 00 00 00 00 40 00 76 & Return addree \\ \hline
01 23 45 67 89 AB CD EF & Saved \%rbx \\ \hline
 & \\ \hline
 & buf = \%rsp \\ \hline
\end{tabular}
\end{center}
B. 
\begin{center}
\begin{tabular}{ |c|c  } 
\hline
00 00 00 00 00 40 00 76 & Return addree \\ \hline
01 23 45 67 89 AB CD EF & Saved \%rbx \\ \hline
35 35 33 32 31 30 39 38& \\ \hline
37 36 35 34 33 32 32 30 & buf = \%rsp \\ \hline
\end{tabular}
\end{center}
C. The program is attempting to return to address 0x040034. The low-order 2 bytes were overwritten by the code for character '4' and the terminating null character. \\ 
D. The saved value of register \%rbx was set to 0x3332313039383736. This value will be loaded into the register before get\_line returns. \\
E. The call to malloc should have had strlen(buf) + 1 as its argument, and the code should also check that the returnd value is not equal to NULL.

\paragraph{Solution to Problem } 3.47\\
A. This corresponds to a range of around $2^13$ addresses. \\
B. $2^6$ = 64 attempts. \\

\paragraph{Solution to Problem } 3.48\\
A. For the unprotected code, we can see that lines 4 and 5 compute the positions of v and buf to be at offsets 24 and 0 relative to \%rsp. In the protected code, the canary is stored at offset 40(line 4), while v and buf are at offsets 8 and 16(lines 7 and 8).\\
B. In the protected code, local variable v is positioned closer to the top of the stack than buf, and so an overrun of buf will not corrupt the value of v.

\paragraph{Solution to Problem } 3.49\\
A. The leaq instruction of line 5 computes the value 8n + 22, which rounded down to the nearest multiple of 16 by the andq instruction of line 6. The resulting value will be 8n + 8 when n is odd and 8n + 16 when n is even, and this value is subtracted from s1 to give s2. \\
B. The three instructions in this sequence round s2 up to the nearest multiplt of 8. \\

\chapter{Processor Architecture}

\chapter{Optimizing Program Performance}
\paragraph{Solution to Problem } 5.1\\
The effect will be to set the value at xp to zero.

\paragraph{Solution to Problem } 5.2\\
We find that for $n<=2$, version 1 is the fastest. Version 2 fastest for $3<= n <= 7$,
version 3 if fastest for $n >= 8$

\paragraph{Solution to Problem } 5.3\\
\begin{center}
\begin{tabular}{ |c|c|c|c|c| } 
\hline
Code & min & max & incr & square \\ \hline
A & 1 & 91 & 90 & 90 \\
B & 91 & 1 & 90 & 90 \\
C & 1 & 1 & 90 & 90 \\
\hline
\end{tabular}
\end{center}

\paragraph{Solution to Problem } 5.4\\
A. In the less optimized code, resister \%xmm0 is simply used as a temporary value, both set and used on each loop iteration. In the more optimized code, it is used more in the manner of variable acc in combine4, accumulating the product of vector elements. The difference with combine4, however, is that location dest is updated on each iteration by the second vmovsd instruction. \\
\begin{lstlisting}
/* Make sure dest updated on each iteration */
void combine3w(vec_ptr v, data_t *dest) {
	long i;
	long length = vec_length(v);
	data_t *data = get_vec_start(v);
	data_t acc = IDENT;

	/* Initialize in event length <= 0 */
	*dest = acc;

	for (i = 0; i < length; i++) {
		acc = acc OP data[i];
		*dest = acc;
	}
}
\end{lstlisting}
B. The two versions of combine3 will have identical functionality, even with memory aliasing. \\
C. This transformation can be made without changing the program behavior, with the exception of the first iteration, the value read from dest at he beginning of each iteration will be the same value written to this register at the end of the previous iteration. Therefore, the combining instruction can simply use the value already in \%xmm0 at the beginning of the loop. \\

\paragraph{Solution to Problem } 5.5\\
A. The function performs 2n multiplications and n additions. \\
B. We can see that the performance-limiting computation here is the repeated computation of the expression $xpwr = x * xpwr$. This requires a floating-point multiplication(5 clock cycles), and the computation for one iteration cannot begin until the one for the previous iteration has completed. the updating of result only requires a floating-point addition(3 clock cycles) between successive iterations. \\

\paragraph{Solution to Problem } 5.6\\
A. The function performs n multiplications and n additions. \\
B. 8 cycles.

\paragraph{Solution to Problem } 5.8\\
A1: 5.00 \\
A2, A5: 3.33 \\
A3: 1.67

\paragraph{Solution to Problem } 5.9\\
\begin{lstlisting}
while (i1 < n && i2 < n) {
    long v1 = src1[i1];
    long v2 = src2[i2];
    long take1 = v1 < v2;
    dest[id++] = take1 ? v1 : v2;
    i1 += take1;
    i2 += (1-take1);
}
\end{lstlisting}

\paragraph{Solution to Problem } 5.10\\
A. It will set each element a[i] to a[i+1], for 0-998 \\
B. It will set each element a[i] to 0, for 1-999 \\
C. In the second case, the load of one iteration depends on the result of the store from the previous iteration. Thus, there is a write/read dependency between successive iterations. \\
D. It will give a CPE of 1.2, the same as for Example A, since there are no dependencies between stores and subsequent loads. 

\paragraph{Solution to Problem } 5.11\\
We can see that this function has a write/read dependency between successive iterations=the destination value p[i] on one iteration matches the source value p[i-1] on the next.

\paragraph{Solution to Problem } 5.12\\
\begin{lstlisting}
void psum1a(float a[], float p[], long n) {
    long i;
    float last_val, val;
    last_val = p[0] = a[0];
    for (i = 1; i < n; i++) {
        val = last_val + a[i];
        p[i] = val;
        last_val = val;
    }

}
\end{lstlisting}

\chapter{The Memory Hierarchy}
\paragraph{Solution to Problem } 6.1\\
\begin{center}
\begin{tabular}{ |c|c|c|c|c|c| } 
\hline
Organization & r & c & $b_r$ & $b_c$ & max($b_r$, $b_c$) \\
\hline
16x1 & 4 & 4 & 2 & 2 & 2 \\
16x4 & 4 & 4 & 2 & 2 & 2 \\
128x8 & 16 & 8 & 4 & 3 & 4 \\
512x4 & 32 & 16 & 5 & 4 & 5 \\
1023x4 & 32 & 32 & 5 & 5 & 5 \\
\hline
\end{tabular}
\end{center}


\paragraph{Solution to Problem } 6.2\\
Disk capacity = 512 x 400 x 10000 x 2 x 2 = 8.182GB
\paragraph{Solution to Problem } 6.3\\
$T_{avg rotation} = 1/2 * T_{max rotation} = 2ms$ \\
$T_{avg transfer} = 0.008ms$ \\
$T_{access} = T_{avg seek} + T_{avg rotation} + T_{avg transfer} = 10ms$ 
\paragraph{Solution to Problem } 6.4\\
The file consists of 2000 512-byte logical blocks.For the disk, $T_{avg seek} = 5ms$, $T_{max rotation} = 6ms, and T_{avg rotation} = 3ms$ \\
A. Best case:  $T_{avg seek} + T_{avg rotation} + 2 * T_{max rotation} = 20ms $ \\
B. Random case: ($T_{avg seek} + T_{avg rotation}$) x 2000 = 16000ms

















\chapter{Linking}
\chapter{Exceptional Control Flow}
\chapter{Virtual Memory}
\chapter{System Level I/O}
\chapter{NetWork Programming}
\paragraph{Solution to Problem } 11.1\\
\begin{center}
\begin{tabular}{ |c|c| } 
\hline
Hex address & Dotted-decimal address \\ \hline
0x0 & 0.0.0.0 \\
0xffffffff & 255.255.255.255 \\
0x7f000001 & 127.0.0.1 \\
0xcdbca079 & 205.188.160.121 \\
0x400c950d & 64.12.149.13 \\
0xcdbc9217 & 205.188.146.23 \\
\hline
\end{tabular}
\end{center}



\end{document}

