\documentclass{report}
\usepackage[utf8]{inputenc}
 
\usepackage{graphicx} %package to manage images
\usepackage[rightcaption]{sidecap}
\usepackage{wrapfig}

 
\usepackage{listings}
\usepackage{color}
 
\definecolor{codegreen}{rgb}{0,0.6,0}
\definecolor{codegray}{rgb}{0.5,0.5,0.5}
\definecolor{codepurple}{rgb}{0.58,0,0.82}
\definecolor{backcolour}{rgb}{0.95,0.95,0.92}
 
\lstdefinestyle{mystyle}{
    backgroundcolor=\color{backcolour},   
    commentstyle=\color{codegreen},
    keywordstyle=\color{magenta},
    numberstyle=\tiny\color{codegray},
    stringstyle=\color{codepurple},
    basicstyle=\footnotesize,
    breakatwhitespace=false,         
    breaklines=true,                 
    captionpos=b,                    
    keepspaces=true,                 
    numbers=left,                    
    numbersep=5pt,                  
    showspaces=false,                
    showstringspaces=false,
    showtabs=false,                  
    tabsize=2
}
 
\lstset{style=mystyle}

\begin{document}

\chapter{A Tour of Computer Systems}

\chapter{Representing and Manipulating Information}

\section{Information Storage}
\paragraph{Solution to Problem } 2.1 \\
A: 0011 1001 1010 0111 1111 1000 \\
B: 0xC97B  \\
C: 1101 0101 1110 0100 1100  \\
D: 0x26E7B5  \\

\paragraph{Solution to  Problem } 2.2
\begin{center}
\begin{tabular}{ |c|c|c| } 
\hline
n & \(2^n\)(decimal) & \(2^n\)(hexadecimal) \\
\hline\hline
9 & 512 & 0x200 \\
\hline
19 & 524288 & 0x80000 \\
\hline
14 & 16384 & 0x4000 \\
\hline
16 & 65536 & 0x10000 \\
\hline
17 & 131072 & 0x20000 \\
\hline
5 & 32 & 0x20  \\
\hline
7 & 128 & 0x80 \\
\hline 
\end{tabular}
\end{center}

\paragraph{Solution to Problem} 2.3
\begin{center}
\begin{tabular}{ |c|c|c| } 
\hline
Decimal & Binary & Hexadecimal \\
\hline\hline
0 & 0000 0000 & 0x00 \\
\hline
167 & 1010 0111 & 0xA7 \\
\hline
62 &  0011 1110 & 0x3E \\
\hline
188 & 1011 1100 & 0xBC  \\
\hline
55 & 0011 0111 & 0x37  \\
\hline
136 & 1000 1000 & 0x88 \\
\hline
243 & 1111 0011 & 0xF3 \\
\hline
82 & 0101 0010 & 0x52 \\
\hline
172 & 1010 1100 & 0xAC \\
\hline
231 & 1110 0111 & 0xE7  \\
\hline
\end{tabular}
\end{center}

\paragraph{Solution to Problem} 2.3 \\
A: 0x5044  \\
B: 0x4FFC  \\ 
C: 0x507C  \\
D: 0xAE  \\

\paragraph{Code} 2-4

\begin{lstlisting}

#include <stdio.h>
typedef unsigned char *byte_pointer;

void show_bytes(byte_pointer start, size_t len) {
	size_t i;
	for (i = 0; i < len; i++)
		printf("%.2x", start[i]);
	printf("\n");
}

void show_int(int x) {
	show_bytes((byte_pointer)&x, sizeof(int));
}

void show_float(float x) {
	show_bytes((byte_pointer)&x, sizeof(float));
}

void show_pointer(void *x) {
	show_bytes((byte_pointer)&x, sizeof(void *));
}

void test_show_bytes(int val) {
	int ival = val;
	float fval = (float)ival;
	int *pval = &ival;
	show_int(ival);
	show_float(fval);
	show_pointer(pval);
}

int main() {
	test_show_bytes(12345);
	return 0;
}

\end{lstlisting}
\includegraphics[scale=0.4]{Solution/images/Code2_4.png}

\paragraph{Solution to Problem} 2.5 \\
A: Little endian:21 \hspace{10mm} Big endian:87 \\
B: Little endian:21 43 \hspace{6mm} Big endian:87 65 \\
C: Little endian:21 43 65 \hspace{1mm} Big endian 87 65 43

\paragraph{Solution to Problem} 2.6  \\
A: 0x00359141 : 00000000001 101011001000101000001 

0x4A564504 : \hspace{1mm} 010010100 101011001000101000001 00 \\
B: 21 \\
C: We find all bits of the integer embedded in the floating-point number, except for the most significant bit aving value 1.

\paragraph{Solution to Problem} 2.7 \\
61 62 63 64 65 66(strlen does not count the terminating null character.

\paragraph{Solution to Problem} 2.9 \\
A: white(black) yello(bule) magenta(green) cyan(red) \\
B: Cyan(011) Green(010) Blue(001)

\paragraph{Solution to Problem} 2.10
\begin{center}
\begin{tabular}{ |c|c|c| } 
\hline
Step & *x & *y \\
\hline\hline
Initially & a & b \\
\hline
Step 1 & a & a\textasciicircum b \\   
\hline  
Step 2 & b & a\textasciicircum b \\
\hline
Step 3 & b & a \\
\hline
\end{tabular}
\end{center}

\paragraph{Solution to Problem} 2.11 \\
A: Both first and last have value k. \\
B: In this case, x and y point to the same location, and a\textasciicircum a = 0 \\
C: In line 4, replace first $\leq$ last to  first $<$ last

\paragraph{Solution to Problem} 2.12 \\
A: x $\&$ 0xFF  \\
B: x\textasciicircum {\textasciitilde 0xFF} \\
C: x \textbar \hspace{1mm} 0xFF \\

\paragraph{Solution to Problem} 2.13 \\
\begin{lstlisting}
int bis(int x, int m) {
	return x | m;
}

int bic(int x, int m) {
	return x & (!m);
}

int bool_or(int x, int y) {
	int result = bis(x, y);
	return result;
}

int bool_xor(int x, int y) {
	int result = bis(bic(x, y), bic(y, x));
	return result;
}

\end{lstlisting}

\paragraph{Solution to Problem} 2.14 \\
\begin{center}
\begin{tabular}{ |c|c|c| } 
\hline
logical operators & bit level operations   \\
\hline\hline  
$\&\&$ & $\&$  \\
\hline  
\textbar \textbar & \textbar  \\
\hline
$!$  & \textasciitilde  \\
\hline
& \textasciicircum \\
\hline
\end{tabular}
\end{center}

\paragraph{Solution to Problem} 2.15 \\
\begin{lstlisting}
!(x ^ y)
\end{lstlisting}

\end{document}